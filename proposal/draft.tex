\newcommand{\zts}[1]{\textcolor{zts}{\textbf{#1}}}

\begin{frame}{问题背景}{多处理器调度问题}

\zts{多处理器调度问题/平行机调度问题}\footnote[1]{Gary M R, Johnson D S. Computers and Intractability: A Guide to the Theory of NP-completeness[J]. 1979.}是最早被广泛研究的 NP-Hard 组合优化问题之一。它研究 $n$ 个作业在 $m$ 台并行相同的机器集上非抢占执行的最小完工时间。带测试调度问题\footnote[2]{Dürr C, Erlebach T, Megow N, et al. Scheduling with explorable uncertainty[C] (ITCS 2018). Schloss-Dagstuhl-Leibniz Zentrum für Informatik, 2018.}
\footnote[3]{Dürr C, Erlebach T, Megow N, et al. An adversarial model for scheduling with testing[J]. Algorithmica, 2020, 82(12): 3630-3675.}
\footnote[4]{Albers S, Eckl A. Scheduling with testing on multiple identical parallel machines[C] WADS 2021, Virtual Event, August 9–11, 2021, Proceedings 17. Springer International Publishing, 2021: 29-42.}是该问题的变体,具有广泛的研究意义。

\begin{probl}[带测试多处理器调度问题]
一个实例 $I$ 包含一组 $n$ 个作业 $J = \{J_1, J_2, \cdots, J_n\}$,每个作业都要在 $m$ 台并行相同的机器集 $M = \{M_1, M_2, \cdots, M_m\}$ 上非抢占执行。每个作业带有一个处理时间上界 $u_j$ 和一个测试操作的时长 $t_j$,如果进行测试,作业将得到一个实际运行时长 $p_j$。每一个作业 $J_j$ 都可以选择在某台机器上花 $u_j$ 时间运行,或者先花 $t_j$ 时间进行测试,之后在同一台机器上运行 $p_j$ 时间。目标是最小化完工时间 $C_{\text{max}}$。
\end{probl}

\end{frame}

\begin{frame}{问题背景}{半在线问题与在线问题}

\begin{itemize}
    \item \textcolor{zts}{\textbf{半在线问题}}:所有作业在时间零时刻到达,记作 $P\;|\;t_j,0\leq p_j\leq u_j\;|\;C_{\text{max}}$。
    \item \textcolor{zts}{\textbf{在线问题}}:调度器需要在作业到达时做出测试决策,并指定机器进行测试或直接执行,每个作业在\textbf{上一个作业的调度决策完成后}到达,记作 $P \; | \; \text{online}, t_j, 0\leq p_j \leq u_j \; | \; C_{\text{max}}$。
\end{itemize}

调度器应当利用作业到达时已知的信息,决定是否对作业进行测试,从而尽可能平衡由于处理时间未知所带来的总时间开销。

\end{frame}

\begin{frame}{问题背景}{竞争比}
    算法的性能可以通过竞争比来衡量。
    \begin{defi}[竞争比]
        假设存在一个确定性算法,令 $C(I)$ 表示该算法在实例 $I$ 上产生的完工时间,$C^*(I)$ 表示最优离线调度的完工时间。定义该算法的竞争比为
        \[
            \alpha=\sup_{I} \frac{C(I)}{C^*(I)}, \quad \text{其中} I \text{遍历所有问题实例,}
        \]
        该算法被称为 $\alpha$-竞争算法。
    \end{defi}

    \zts{注:} 对于带测试多处理器调度的离线最优算法,可以视为每个工作的处理时间为 $u_j'= \min\{u_j, t_j + p_j\}$ 的普通多处理器调度问题。
\end{frame}

\begin{frame}{问题背景}{现有研究成果-通用测试实例}

\zts{通用测试实例}:对于所有测试操作时间都为单位时间的问题 $P\;|\;\text{online}, t_j=1,0\leq p_j \leq u_j\;|\; C_{\text{max}}$。
    \begin{itemize}
        \item 当只有一台机器时,Dühr等\footnote[1]{Dürr C, Erlebach T, Megow N, et al. Scheduling with explorable uncertainty[C] (ITCS 2018). Schloss-Dagstuhl-Leibniz Zentrum für Informatik, 2018.}
\footnote[2]{Dürr C, Erlebach T, Megow N, et al. An adversarial model for scheduling with testing[J]. Algorithmica, 2020, 82(12): 3630-3675.} 提出,当 $u_j \leq \varphi := \frac{\sqrt{5}+1}{2}$ 时对作业 $j$ 进行测试,得到 $\varphi$-竞争算法;或者以概率 $f(u_j) = \max \left(0, \frac{u_j(u_j-1)}{u_j(u_j-1)+1}\right)$ 概率对作业进行测试,得到 $\frac43$-竞争的随机化算法。可以证明\footnote[3]{Albers S, Eckl A. Scheduling with testing on multiple identical parallel machines[C] WADS 2021, Virtual Event, August 9–11, 2021, Proceedings 17. Springer International Publishing, 2021: 29-42.},这两种算法是最优的,即 $\varphi$ 是确定性算法的竞争比下界,$\frac43$ 是任何随机化算法的期望竞争比下界。

\item 当有 $m\geq 2$ 台机器时,存在 $\varphi(2-\frac1m)$-竞争算法
\footnote[4]{Graham R L. Bounds for certain multiprocessing anomalies[J]. Bell system technical journal, 1966, 45(9): 1563-1581.}。

\item 对于任何确定性算法,\textbf{竞争比下界为 $2$}。(构造实例用了两台机器、三个工件)\footnote[5]{Albers S, Eckl A. Scheduling with testing on multiple identical parallel machines[C] WADS 2021, Virtual Event, August 9–11, 2021, Proceedings 17. Springer International Publishing, 2021: 29-42.} 
    \end{itemize}
    
\end{frame}

\begin{frame}{问题背景}{现有研究成果-一般测试实例}
    在 Gong 的论文\footnote{Gong M, Chen Z Z, Lin G, et al. Randomized algorithms for fully online multiprocessor scheduling with testing[J]. arXiv preprint arXiv:2305.01605, 2023.} 中,证明了任何\textbf{确定性}算法的竞争比下界为 2.2117。
    
    同时,论文还提出了一种确定性算法集成的随机化算法,其期望竞争比为 $\frac{3\varphi+3\sqrt{13-7\varphi}}{4}\approx 2.1839$。

    \pic{0.6}{pic/bounds}
\end{frame}

\begin{frame}{主要研究内容}{}

    探索并证明在线带测试的多处理器调度问题的\textbf{确定性算法}竞争比下界。即:证明不存在竞争比总是低于某个下界的确定性算法。

    \[
        P \; | \; \text{online}, t_j, 0\leq p_j \leq u_j \; |\; C_{\text{max}}
    \]

    对该问题进行程序建模,通过分析特定情况下的竞争比表现,识别出影响竞争比的关键因素。在此基础上,我还将对现有算例进行分析,特别关注那些导致竞争比达到最坏的特殊情况。运用计算机辅助的方法在\textbf{有限}的时间内搜索出更广泛的实例,以验证该问题的竞争比下界,并通过严谨的分析来确认最终结果。
\end{frame}

\begin{frame}{研究意义和目的}{}
\begin{enumerate}
    \item 多处理器调度问题作为一个重要的组合优化问题,其\textbf{在实际生产环境中具有广泛的应用场景},如工业生产和资源分配等领域。

    \item \textbf{带测试的多处理器调度问题},引入了相同的工作的不同处理方式(一种未知效果的新处理手段),并以提高整体系统的效率为目标,同样有着重要的研究意义。

    \item 目前在线带测试多处理器调度问题竞争比下界还有较大的提升空间。现有的研究主要集中在\textbf{机器数量、工作数量较少的情况下,使用手动构造最坏情况}来分析竞争比下界,但对于更一般的情形,特别是当 $m \geq 3, n\geq 4$ 时,现有的竞争比下界仍有改进余地。

    \item 通过引入计算机辅助方法,利用程序设计方法\textbf{更高效地构建和分析决策树},可以扩大搜索范围,有希望进一步提高竞争比下界。
\end{enumerate}

\end{frame}

\begin{frame}{最优化模型和搜索框架}{}
    同时我还将参照朴素平行机调度问题\footnote{Gormley T, Reingold N, Torng E, et al. Generating adversaries for request-answer games[C]//Proceedings of the eleventh annual ACM-SIAM symposium on Discrete algorithms. 2000: 564-565.} 数学模型,设计\textbf{带测试的多处理器调度问题的竞争比下界求解模型}。

    竞争比下界求解可以看成是一个 max-min 的优化问题。
    % 通过寻找每种决策的最坏数据来求解竞争比的下界。

    \begin{enumerate}
        \item 设决策序列集合为 $A$,实例集合为 $\mathcal{I}$,则最坏情况下的竞争比可表示为 $\alpha^* = \sup_{I \in \mathcal{I}} \alpha(I)=\sup_{I \in \mathcal{I}}\min_{a\in A}\frac{C(I,a)}{C^*(I)}$。
        \item 将最优化问题转为判定性问题:给定 $\alpha_0$,判断是否存在实例 $I^*$ 使得 $\alpha(I)\geq \alpha_0$,即 $\min_{a\in A} \frac{C(I,a)}{C^*(I)}\geq \alpha_0$,即 $\forall a\in A, C(I,a)\geq \alpha_0C^*(I)$。
        \item 不等号左右两端都是 $I$ 的分量的线性组合及其 min 和 max。问题转化为一个或多个线性规划问题,可以通过加以剪枝的搜索方法来构造每个分量的值。
        % 可以使用单纯形法等多种经典算法求解,
    \end{enumerate}
\end{frame}

\begin{frame}{研究计划进度安排及预期目标}{}
\zts{进度安排}
    \begin{enumerate}
        \item 2025.2.28 以前,完成问题建模和文献调研,归纳现有方法。完成三合一等准备工作。
        \item 2025.3.15 以前,设计计算机搜索的代码并初步验证现有研究成果。
        \item 2025.3.31 以前,通过进一步的理论分析,确定搜索剪枝策略。
        \item 2025.4.15 以前 对剪枝策略和现有运行结果,进行深入理论分析和证明,进一步改进竞争比下界。
        \item 2025.5 以前,撰写毕业论文,持续尝试改进搜索策略,以得到更好的结果和获得更快的程序验证效率。
    \end{enumerate}
\zts{预期目标}
    \begin{enumerate}
        \item 通过广泛实例模拟,验证现有研究成果中的竞争比下界。
        \item 通过计算机进行高效剪枝搜索,寻找更大的竞争比下界。
    \end{enumerate}
\end{frame}